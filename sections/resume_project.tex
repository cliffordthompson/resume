%***************************************************************************
%
%  Copyright (c) 2003-2020 Clifford Thompson
%
%  This program is free software: you can redistribute it and/or modify
%  it under the terms of the GNU General Public License as published by
%  the Free Software Foundation, either version 3 of the License, or
% (at your option) any later version.
%
%  This program is distributed in the hope that it will be useful,
%  but WITHOUT ANY WARRANTY; without even the implied warranty of
%  MERCHANTABILITY or FITNESS FOR A PARTICULAR PURPOSE.  See the
%  GNU General Public License for more details.
%
%  You should have received a copy of the GNU General Public License
%  along with this program.  If not, see <http://www.gnu.org/licenses/>.
%
%**************************************************************************

\resumesection{\hypertarget{research}{Selected Projects}}

\resumeproject
{Intrument Procedure Development System}
{Various}
{October 2007 - Present}
{IPDS is a software product that will be used by the Federal Aviation Administration to automate the design of airplane landing procedures for airports throughout North America. Currently, I am completing the leading of 3-6 developers through 800 person days of work. Previously, I was responsible for the development of the web services component which allows the IPDS application to interact with the FAA's remote aviation databases. The system is being developed using C++ and various XML technologies (Schema, XLST, XPath, Web Services).}

\resumeproject
{NCOT - Submarine Fire Control System}
{T. Skaling, C. Thompson, K. Pak, R. Young, J. DeKleine, L. Trevorrow, J. Sabbagh, K. Power}
{October 2004 - October 2007}
{NCOT is a combat simulator for the operations room equipment of the various Canadian Naval ships. The system is realized as a multi-processed and networked simulation which can span up to 50 machines. In particular, I was tasked with the simulation of the Submarine Fire Control System, and was involved in the requirements elicitation, software design, implementation, and testing. %The Fire Control System %has one of the lowest defect rate per SLOC and highest SLOC per developer day of any team in the
%NCOT group.
I also lead the porting and regression testing of the Submarine Fire Control System from UNIX to Linux. Development in this project was done largely in ANSI C,  with cross-compiled portions in C++.}

%\resumeproject
%{IPDS - Web Services Component}
%{C.Thompson}
%{October 2007 - Present}
%{IPDS is a software product that will be used by the Federal Aviation Administration (FAA) to automate the design of airplane landing procedures for airports throughout North America. I am currently leading the development of the web services component which will allow the IPDS application to interact with the FAA's remote aviation databases. This component is being developed using ANSI C and C++ cross-compilation, and various XML technologies (XSLT, XPath, XMLSchema).}


%\newpage
%\resumesection{Selected Research and Projects}

%\resumeproject
%{TAD Radio M-Series Programmer}
%{C. Thompson}
%{October 2003 - April 2004}
%{The M-Series Programmer allows TAD technicians/dealers to program various models from TAD's M-Series line of VHF-band radios on all Windows-based machines (Win9x - WinXP). Rigorous software development methodology was used throughout the project including the \emph{requirements elicitation} \& \emph{definition}, \emph{software design}, \emph{implementation}, \emph{quality testing}, and \emph{deployment}. Borland C++ Builder was used for implementation, and CVS for version control. Various serial/parallel communications protocols were designed \& implemented.}

%{Under a 6 month contract with TAD Radio of Canada, I took the lead development role for the M-Series Programmer project.
%The M-Series Programmer allows TAD technicians/dealers to program various models from TAD's M-Series line of VHF-band radios on all
%Windows-based machines (Win9x, WinMe, WinNT, Win2000, \& WinXP). Rigorous software development methodology was
%used throughout the project including the \emph{requirements elicitation} of management, employees, and technicians of
%TAD, \emph{requirements definition}, \emph{software system design}, \emph{implementation} (including some code re-use
%from legacy software), \emph{system testing}, and \emph{software deployment}.

%\vspace{0.2 cm}

%Borland C++ Builder was used as the primary development IDE, and CVS was used for version
%control. The software was designed to allow for ease of expansion as future M-Series radios are
%developed, and porting to other platforms, such as Linux. Various communication protocols were designed/implemented (both proprietary and XModem)
%for communication over both serial and parallel ports. Technical documentation was also generated for the latest M-Series radio model.
%Mr. Darryl Schultz can be contacted at TAD Radio for more information on this project.}

%\resumeproject
%{Master of Science Thesis}
%{C. Thompson}
%{May 2006}
%{My thesis presents a new theory of memory modelling in natural language processing that attempts to
%address the problem of growing complexity. The theory proposes a more general model where multiple semantic
%representations are used to model the observed behaviour of working (short-term) and long-term memory.
%Prolog was used for testing and implementation.}

%\resumeproject
%{UNBC Language Engine}
%{Dr. Charles Brown, N. Hagen, C. Thompson}
%{September 2000 - April 2001}
%{I worked with Dr. Charles Brown and a fellow student to develop a grammar for determiners, quantifiers, and
%relative clauses for the UNBC's LangEng Project. In addition, we performed a literature review on semantic
%structures for determiner and quantifier phrases.}

%\resumeproject
%{Taxi Dispatching System}
%{R. Wilson, N. Martin, N. Hagen, C. Thompson}
%{September 2000 - April 2001}
%{Our team designed and implemented a client/server taxi dispatching system using rigorous software development
%methodology. Rational Rose was used for the design phrase. C++ and the Unix QT Library Tools were used to implement
%the backend and GUI.}

%\resumeproject
%{Academic AI Project - 8 Puzzle}
%{C. Thompson}
%{January 2000 - April 2000}
%{In the 8-puzzle project, I create a heuristic search strategy that combined four smaller heuristics (Manhattan
%distance, plus three that I created) and reduced the search space from ~23,000 nodes with breadth-first search down to 69
%nodes for a 16-move starting configuration. The project was implemented using SWI Prolog and C++.}

%\resumeproject{Personal AI Project - Boggle Puzzle}
%{Independent Project}
%{July 2004}
%{This program finds all possible solutions for a Boggle puzzle configuration, given a
%dictionary. Breath-first search was used in combination with heuristic pruning strategies
%resulting in an extremely efficient search algorithm. The program was implemented in using only
%pure C++ (no STL) and the C standard library, and executes on both Windows and Linux platforms.}

%\resumeproject
%{AI Project - Path Finding with Autonomous Robots}
%{C. Thompson}
%{January 2000 - April 2000}
%{This research paper investigated the \emph{state-of-the-art} of dynamic path finding algorithms, map representations,
%and obstacle avoidance for autonomous agents. Implementations were performed using C++ to test the effectiveness of the various
%algorithms in differing situations.}

%\resumeproject
%{Data Mining/Warehousing Research Report}
%{C. Thompson, N. Gruzling}
%{October 2000 - November 2000}
%{In this research report, our group investigated the current state of the art of data mining and
%data warehousing techniques. A copy of the report is available at my website.}

%\resumeproject
%{DEC PDP-8 Implementation}
%{N. Hagen, C. Thompson}
%{January 1997 - April 1997}
%{Our team was the first team in UNBC history to complete a fully functional replica of the PDP-8.
%With the computer functional, we were able to write a program to calculate the Fibonacci number sequence
%using PDP8 machine language.}

\blfootnote{$^{\dagger}$ References are available upon request.}
